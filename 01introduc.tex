% Prof. Dr. Ausberto S. Castro Vera
% UENF - CCT - LCMAT - Curso de Ci\^{e}ncia da Computa\c{c}\~{a}o
% Campos, RJ,  2022
% Disciplina: An\'{a}lise e Projeto de Sistemas
% Aluno: Luiz Miguel Guedes Gomes
 
\chapterimage{sistemas.png} % Table of contents heading image
\chapter{ Introdu\c{c}\~{a}o}

\textit{ O Sistema Pollinator} \'{e} um sistema destinado a modernização e gestão do setor de com\'{e}rcio de plantas e flores, tornando-o mais eficiente e satisfatório tanto para o cliente, quanto paro o gestor do negócio, gerando para este último um melhor direcionamento em relação à organização de seu negócio, através de dados como: Orçamento disponível para encomenda de mercadorias e reinvestimento na estrutura do negócio, ao descobrimento de tendências no mercado para determinado período, ao controle de perda de mercadorias, à melhor especificação da ordem de pedido dos clientes e de encomenda aos fornecedores, entre outras funções, levando assim, a um maior lucro e desenvolvimento empresarial.  As refer\^{e}ncias bibliogr\'{a}ficas b\'{a}sicas a serem consultadas s\~{a}o: \cite{Dennis2014}, \cite{Dennis2019} \cite{Gane1983} e \cite{Sommerville2011}. Como bibliografia complementar ser\~{a}o considerados: \cite{Satzinger2012}, \cite{Shelly2012}, \cite{Valacich2020}, \cite{Kendall2020}, \cite{Budgen2021} e \cite{Engholm2013}.

Neste documento apresentamos, passo a passo,  as atividades relacionadas com a An\'{a}lise e Design do sistema Pollinator


 \section{Descri\c{c}\~{a}o do Sistema Computacional a desenvolver}

        \subsection{abcde}


        \subsection{defgh}

 \section{Identificando as componentes do sistema Pollinator}


     \subsection{Componente: Hardware}
	 \begin{outline}
           \1 Computadores por unidade
	 	\2  Dois em caixas
		\2  Três em escritórios 
		%\2 - Um para monitor central 
	\1 Tablets por unidade
		 \2 Cinco distribuídos pelos setores
	\1 Monitores %por unidade
		%\2 Um para cada computador
		%\2 Um monitor central
	\1 Smart TV de 40 polegadas
	\1 Estabilizadores	
	\1 Nobreaks 
	\1 Switchs
	\1 Filtros de linha
	\1 Roteadores 
	\1 Equipamentos e pacotes de internet
	\1 Sistema antifurto
	\1 Relógios de pontos
	\1 Servidores
		\2 Servidor de hospedagem de sites
		\2 Servidor de banco de dados
	\1 Ar condicionado para a sala de servidores
	\1 Impressoras
		\2 Para os escritórios
		\2 Para os caixas
	\1 Mobília
	\1 Câmeras
	\1 Sensores
	\1 Grades 
	\1 Cabos de energia 
	\1 Relógio de ponto
	\1 Adaptações nas instalações elétricas 
	\1 Cabos de ethernet 
	\1 Portas automáticas
	\1 Recursos para a acessibilidade 
	 \end{outline}

     \subsection{Componente: Software}
	\begin{outline}
	\1 POLLINATOR %Still choosing names
		\2 Website
		\2 Aplicativo
		\2 Sistema de gerenciamento
			\3 Estoque
			\3 Vendas
			\3 Encomendas
			\3 Entregas
			\3 Preços
			\3 Promoções
			\3 Informações sobre produto
			\3 Finanças
			\3 Fornecedores
			\3 Clientes
			\3 Usuários
			\3 Avaliações e comentários
			\3 Orçamento
	\1 Sistemas operacionais
	\1 Pacotes Office
	\1 Antivírus
	\1 Gerenciador de banco de dados
	\1 Acessibilidade 
	\1 Sistema de versionamento %Diagramas UML
	\1 Editor de linguagens de desenvolvimento	
	 \end{outline}
     \subsection{Componente: Pessoas}
	\begin{outline}
	\1 Usuários
	\1 Clientes
	\1 Funcionários da empresa
	\1 Fornecedores 
	\1 Administrador financeiro da empresa
	\1 Técnicos de rede
	\1 Técnicos em eletroeletrônica
	\1 Operador de banco de dados
	\1 Arquitetos de software
	\1 Analista de projeto
	\1 Gerente de segurança
	\1 Gerente de projeto
	\1 Desenvolvedores
	\1 Designers
	\1 Equipe de instalação/manutenção
	\end{outline}
     \subsection{Componente: Banco de Dados}
	\begin{outline}
	\1 Base de dados
		\2 Catálogo de produtos
		\2 Produtores
		\2 Fornecedores
		\2 Informações sobre os produtos
		\2 Estoque
		\2 Preços
		\2 Vendas
		\2 Clientes
		\2 Funcionários 
		\2 Encomendas dos clientes
		\2 Encomenda aos fornecedores
		\2 Avaliações e comentários
		\2 Entregas
		\2 Orçamento
	\1 Gerenciador de banco de dados
	\1 Servidores de banco de dados
	\end{outline}
     \subsection{Componente: Documentos }
	\begin{outline}
	\1 Notas fiscais
	\1 Orçamentos
	\1 Documentação descritiva do sistema
	\1 Relatórios
	\1 Manuais do sistema  
	\1 Diagramas UML
	\end{outline}
     \subsection{Componente: Metodologias ou Procedimentos}
	\begin{outline}
	\1 An\'{a}lise do sistema
		\2 An\'{a}lise do sistema antigo que será substituido
		\2 Construção do plano de desenvolvimento e instalação do sistema
		\2 Estudo detalhado da interação entre componentes do sistema
		\2 An\'{a}lise dos requisitos dos computadores, servidores e tablets
		\2 An\'{a}lise do sistema de hierarquia entre usuários do sistema
	\1 Levantamento de requisitos
		\2 Listagem dos recursos necessários para funcionamento do novo sistema
	\1 Formação de equipe
		\2 Contrato de novo pessoal para equipe 
		\2 Treinamento e capacitação de novos contratados direcionados ao projeto
		\2 Capacitação de antigos funcionários e gerentes ao novo sistema
	\1 Implementação do sistema
		\2 Desenvolvimento do aplicativo, sistema interno e website
		\2 Desenvolvimento do programa de análise de dados e gerador de orçamento e apontador de tendências e perdas de mercadoria
		\2 Desenvolvimento do programa de encomendas personalizadas
		
		\2 Montagem dos servidores e roteadores de rede
		\2 Montagem dos equipamentos (computadores, monitores, cabos, tablets, bancadas de tablets, etc.)
	\1 Testes do sistema
		\2 Testes privados
		\2 Testes internos a equipe
		\2 Testes públicos
	\1 Manutenção do sistema
		\2 Treinamento da equipe de manutenção
		\2 Backups periódicos 
		\2 Utilização de sistema de versionamento
		\2 Manutenção preventiva
		\2 Manutenção preditiva
		\2 Manutenção corretiva caso necessária
	\end{outline}
     \subsection{Componente: Mobilidade}
	\begin{outline}
	\1 Roteadores de rede
	\1 Aplicativo
	\1 Tablets e smartphones para auxiliar os funcionários no antendimento
	\1 Notebooks para gerentes 
	\1 P\'{a}gina web
	\end{outline}
     \subsection{Componente: Nuvem}
	\begin{outline}
	\1 Hospedagem de servidores em nuvem de forma auxiliar  
	\1 Armazenamento de dados importantes em nuvem
	\1 Versionamento do projeto na nuvem
	\end{outline}
